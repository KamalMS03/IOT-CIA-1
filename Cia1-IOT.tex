\documentclass[nobib]{MSword}
\usepackage[english]{babel}
\usepackage{csquotes}
\usepackage{lipsum}
\title{Improving IOT Based Architecture of Healthcare System- Review}
\author{Kamalnath M S}
\date{22 January 2023}
\begin{document}
\maketitle
\section{Introduction}
This paper deals about advances in the IOT Engineering just as framework structure for medicinal service frameworks. like mHealth and 6LoWPAN based designs can investigated. Multisensor based
framework plans that sense blood glucose,wireless temperature,
pulse and Electrocardigram heart (ECG) and so on are
investigate in this paper.The primary inspiration of this paper is to give a original IoT design sustaining
adaptability, adaptation to non-critical failure and human
services administrations dependent on tweaked 6LoWPAN for
medicinal conditions. 
\section{Existing System Architecture}
IPV6 convention needs vast measure of handling force and transfer speed. The data is been established with 6LoWPAN all the data are been transfer to IEEE 802.15.4 to the gateway. It changes the entry for the groups to IPV6 distributes and also send to ordinary IPV
framework to the server for getting ready to data to flow in the
system.
\begin{figure}
\includegraphics[width=0.5\textwidth]{esa.jpg}
\caption{Existing System Architecture}
\end{figure}
\clearpage
\section{Proposed System Architecture}
The system has cloud server which is collecting data then that data is being pass to internet server then with the help of wifi is connected through a gateway which transfer data to SoC single board then with help of sink node the data further transfer to 6LoWPAN network then it transfer data to star-based 6LoWPAN medical sensor and this entire system or architecture is being attach to the web client with help of mobile,tablet,laptop they can easily access the data and see all the data which they require.



\includegraphics{esa1.jpg}
\newline
\section{Wireless System Design For Healthcare System}
\textbf{Wireless Temperature System:} Remote sensor structure and Raspberry pi based wireless temperature watching framework is prescribed that helps to notice the ace when the temperature outflanks a specific range.
\newline
\newline
\textbf{Wireless ECG Monitoring System:} The ECG reading from various
customers, evaluated using the wireless system are the traded through Zigbee to server. Then , the server is accountable for further results.
\newline
\newline
\section{Sensor based Architecture}
\textbf{ECG Sensor:}Sensors are used to perceive and check the electrical activities inside the heart.These sensors can be used for ECG watching system. 
\newline
\newline

\textbf{Heart rate sensor:}The heart rate sensors helps to collect heart rate data for performing datasets.
\newline
\newline
\includegraphics{esa2.jpg}
\section{Conclusion}
The sensor architecture shows how how the sensors work in a healthcare system and can improve the working in the healthcare sector.

\end{document}